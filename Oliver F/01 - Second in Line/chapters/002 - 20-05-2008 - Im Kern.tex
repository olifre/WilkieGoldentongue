\chapter{Im Kern}
\label{cha:im-kern}
Das Licht von Fackeln schien mir entgegen. Alles schien so, wie erwartet, nicht die geringste Abweichung von der Simulation. Und doch --- irgendetwas war anders. Ich konnte es nicht greifen, es war mehr ein Gefühl als ein sicherer Gedanke.

Die Höhle erschien trotz des Fackellichtes sehr künstlich und irgendwie modern. Metallisch glänzende, perfekt runde Wände, eine passende Decke und ein ebensolcher Boden waren hinter dem zu erkennen, was erst in den letzten Jahren (oder Jahrzehnten?) hinzugekommen sein musste. Felle, diverse Ornamente, Vasen, Krüge, antikes Geschirr und jegliche andere Art von Kunstwerk, zu der diese Zeit fähig war. Mit viel Liebe hatten die Bewohner einen Altar in der Mitte des Raumes geschaffen, kreisrund, und nachdem sich meine Augen nun langsam an das Licht gewöhnt hatten, konnte ich auch die Personen erkennen: Ein Priester mit einer Art Stola aus Fell kniete vor dem Altar. Seine Züge waren eingefallen, er war vom Alter gezeichnet, doch man sah ihm an, dass er nicht nur weise, sondern auch stark war. 

Es würde nicht leicht werden.
Die zwei Wachen, welche sich rechts und links im Raum aufhielten, trugen einfache Speere, mit handgeschärften Steinen als Spitzen. Mit entschlossenen, aber leicht neugierigen Blicken blinzelten sie zu mir herüber, dann sahen Sie wieder zum Altar und zum Eingang hinter mir. Nicht das geringste Anzeichen von Angst war in ihren Gesichtern auszumachen.

Ich schloss die Augen, nur um wieder die gleiche Szene vor meinem inneren Auge zu sehen. Es war die Simulation, und ich sah, wie der Hohepriester der Apalachi über den Altar strich. Vorsichtig versuchte er, etwas zu ertasten, und nahe an der rechten Kante musste er gefunden haben, was er suchte. Eine zuckende Fingerbewegung gefolgt von weiteren Bewegungen verriet, dass es eine Art Codeschloss war. Die Simulation jedoch konnte die Einzelheiten nicht genauer erfassen.

Ich öffnete die Augen wieder. Wir brauchten ihn. Wir brauchten ihn, damit wir existierten, damit wir hier sein konnten. Mir wurde beim Gedanken daran ein wenig schwindlig, doch es war real. Realer als ich selbst vielleicht\dots{}

Keine Zeit für Gedanken. Einen kurzen Moment brauchte ich, um mich wieder zu fassen, dann ging ich vorsichtig, aber selbstbewusst auf den Priester zu. Er erhob sich langsam, denn ihm war meine Ankunft selbstverständlich nicht entgangen. Würdevoll drehte er sich um, und mit festem Blick schaute er mir in die Augen.

Es waren nur wenige Augenblicke, die allerdings wie eine Ewigkeit erschienen. 
\enquote{Willkommen, Fremder}, begann der Priester schließlich. Ich besann mich gerade noch rechtzeitig darauf, dass es der Hohepriester dieses Stammes war, und wollte niederknien, doch er fasste meine Schulter und hinderte mich daran.

Im Moment der Berührung hatte ich ein seltsames Gefühl; ein fast elektrisches Knistern, welches von der Schulter ausging, erfasste jedes Molekül meines Körpers, ließ mich erzittern und schließlich vor Angst erstarren. Es war\dots{} unbekannt. Es gehörte nicht hierher. Nicht in diese Zeit, nicht zu diesem Hohepriester\dots{}

Ich hob langsam den Kopf, während immer noch ein Frösteln meinen Rücken hinablief. Er sah mich an --- nein, er starrte --- bohrte. Meine Gedanken liefen durcheinander, der Sprung, das Wissen um das, was sein könnte und für mich schon war, alles, was ich erlebt hatte, oder erlebt haben könnte, war gleichzeitig da, und unkontrollierbar.
Mit einem Moment war das Zittern verschwunden, es war nur eine normale Hand auf meiner Schulter, und ein Hohepriester der Apalachi, der vor mir stand.

\enquote{Fati und Hata, ihr könnt gehen. Ich habe diesem Mann auf Befehl des Häuptlings etwas Wichtiges zu zeigen.} Kaum hatte der Hohepriester diese Worte gesprochen, hörte ich, wie die Wachen verschwanden, nachdem sie einen Moment lang darüber nachgedacht haben mussten, was eigentlich los war. Doch gehorsam waren sie, gehorsam und unterwürfig --- aber kampfbereit. 
Nicht einen Moment hatte der Hohepriester den Blick von mir gewendet, und als die Schritte der Wachen verklungen waren, begann er, zu sprechen.

\enquote{Tariq und Toka nennt ihr euch also. Ich habe euch bereits erwartet. Ihr sucht den Stein, der dieser Höhle eine andere Zeit gibt.}
Ich starrte ihn an, und für kurze Zeit mischte sich ein Ausdruck des Entsetzens auf mein Gesicht. Er \emph{wusste} es. War es nur eine reine Annahme? Unsere Namen musste er vom Häuptling erfahren haben, vielleicht unterstellte er diese Absicht jedem fremden Besucher. Sein faltiges Gesicht, welches mir aus wissenden Augen entgegensah, sagte jedoch etwas anderes. Eine leichte Erregung, aber keine Angst war ihm anzusehen. Aus der Nähe wirkte sein langer, grauer Bart einschüchternd, seine wilde Haarpracht mächtig und das blaue Gewand irgendwie königlich.

\enquote{Ich lebe in allen Zeiten, und ich kenne Euch. Auch Ihr kennt meinen Namen.} 
Man konnte förmlich hören, wie er mich betont mit großen Buchstaben ansprach, als sei ich ein Diener eines fernen Königs. In gewisser Weise stimmte das sogar, nur, dass es zu dieser Zeit keine Könige gab, und diese Sprache nicht üblich war\dots{}

\enquote{Wir sind uns bereits einmal begegnet, doch Ihr werdet euch nicht erinnern --- noch nicht. Was hat Euch die Herrscherin für den Austausch mitgegeben?}

\enquote{Den Austausch?}, fragte ich zögerlich. Natürlich sollten wir eine Nachbildung gegen den Stein tauschen --- wir, das heißt, ich --- aber es sollte \emph{unbemerkt} geschehen. Er \emph{konnte} es einfach nicht wissen.

Der Hohepriester, so es denn einer war, seufzte laut. \enquote{Sir, vielleicht ist es in beiderseitigem Interesse, wenn Sie mir jetzt den mit Polyresin ummantelten Bleiklotz jetzt geben. Das sollte die ganze Prozedur ein wenig abkürzen und erspart Ihnen den Ärger, heimlich vorzugehen.}

Als er begonnen hatte, zu sprechen, hatte es in meinem Ohr kurz gepiepst. Er hatte die Sprache gewechselt und modernes Englisch benutzt. 
Ein Schauder lief langsam durch meinen Körper, und ich trat einen Schritt zurück. Noch nicht einmal das übliche krächzende \enquote{Was\dots{}?}, das nun laut Drehbuch kommen sollte, brachte ich hervor. Vielleicht, weil es kein Film war, und ich kein Schauspieler\dots{}

Das Kribbeln --- er musste mich eben gescannt haben. Er musste\dots{}
\enquote{\dots{} aus der Zukunft sein? Nein, das bin ich nicht. Ich habe Euch gesagt, ich lebe in allen Zeiten. Können wir nun?}
