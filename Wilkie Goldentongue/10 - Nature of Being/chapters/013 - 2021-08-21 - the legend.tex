\chapter{The Legend}
\label{cha:the-legend}
\enquote{I have already told Aurora a bit about our background, but held off on all the questions about our current situation --- and magic}, Simetra started, addressing both of her companions and relying on communication by thought again to train this new skill. \enquote{Aurora told me we will certainly keep going in this sphere for quite a while longer, so we can make good use of the time. Would you mind if I start with some questions which have been weighing on my mind last night?}
Apollo shook his head, so Simetra continued, facing mostly Aurora. \enquote{I have been watching this strange forest through the night, and I got the impression that this dead, unnatural forest has but one purpose: Collecting magical energy. Is this correct?}

\enquote{I'm surprised you got to the bottom of this evil concoction so fast, after having learned about magic yourself just a short while ago}, Aurora confirmed. \enquote{This is a violation of how nature works: A vile device to extract mana, the magical energy, from all living beings and the land itself. All the creatures living on the land before have already been sucked dry of their energy, and only little remains which is pulled from the earth by these tree-like structures. You may have seen the flow of the energy at night --- it is now almost negligible, and most of the energy is spent in the maintenance effort these mechanisms you have encountered are performing. All life is now gone from these lands.}

\enquote{Does this mean we are safe, as long as we do not leave any trace in the forest below? I have dropped a piece of fluff from my feathers last night, and it was absorbed by the trees, which made me understand what this forest seems to be doing. We must have been easy to track when we were still walking on the ground before.}
Aurora closed her eyes shortly before she responded. \enquote{Yes, I think we are safe --- even though we seem to be flying straight in a single direction at first glance, we have actually changed direction several times in the past hours, zig-zagging our way towards the target. So I believe we can not be followed that easily, even in case they would manage to trace the fluff you dropped before and connect it to us. I mentioned this harvesting violates the ways of nature: It is not only true because they deplete all living things and the earth itself, but this technique also does not differentiate between different kinds of mana. Both of you seem to be able to see the magical aura of living things, and you must have noticed different colours and feelings radiated by different beings. For the mana extracted by these trees, though, it is neutral, bereft of any colour.}

Apollo finally interjected: \enquote{How can you pinpoint our location and direction in this ocean of trees? And who is behind this mana collection, what has happened to the land which was here before? We've had visitors from outside town, how did they ever get through this deathly trap?}

\enquote{You are right, I should probably start from the beginning. Before I tell you the legend which has brought me here, let me answer the easier of your questions: We use ley lines to find out our location, so we essentially navigate by magic. Since I am not home in these lands, this only roughly tells me where we are, and in which direction we are going. It requires lengthy training to notice the natural flow of magic, and see the ley lines, but it is natural to us fairies.

  I already told Simetra before that there is a legend passed down among the fairies, which is what brought me here --- and this legend is connected to your story, I believe.

  This legend is actually both history and future at the same time. We live for many hundreds of years, and have had the chance to observe regular patterns in history --- and this is one of them. Over a period of 500 years, as if it is a natural process, the alignments of good and evil intensify. While there are many different creatures which all have their very own characters, and the overall, natural balance is kept, it happens that more extreme evil lords and good heroes evolve. This is hard to put into words, since good and evil are concepts which are different for everyone --- maybe I should define that a \enquote{good} force wants to preserve the world as it is, and ease life for everybody, while an \enquote{evil} force wants to destroy the world we know, and extinguish all life upon this planet. This means both \enquote{good} and \enquote{evil} are longing for power, and over this time span of 500 years, this forceful fight is intensifying, until the forces finally clash and the world is reset by nature itself. This is the grand picture, and the legend I am about to tell you is essentially a recollection of the major incidents in one such cycle. There are common events in all the cycles, and it seems the underlying mechanics lead to similar characters evolving and clashing against each other. Please keep this in mind when listening to my legend --- it is a story we learn about as children, and we experience this cycle several times in our lives.}

\froufrou{}

It was once a human king and his wife, who lived and ruled the kingdom of Amazonia. King Ephraistos was a wise and calm ruler, who planned long into the future to guarantee his people a safe and comfortable life. His wife, Seraphina, supported him with all her might, and carefully observed society to identify if anything was amiss, or discontent was developing in their people. While the kingdom was a monarchy, it was ruled in the interest of the people, and there were rarely any complaints or unhappiness in the societies of Amazonia.

While Ephraistos and Seraphina had been married for political reasons, they had come to love each other and shared their happiness. They bore two children: The firstborn, a courageous boy called Prinos, was not much of a born ruler, but shared the sense of justice and the love for their nation with his parents. The second born was a graceful daughter by the name of Augustine. She lived more of a secluded life, spending her days with books and studying the nature of magic, only making public appearances whenever it could not be avoided. For this reason, not much was known about her, but an aura of cleverness and intelligence seemed to be hidden behind the grace she showed openly. The royal family also kept a familiar, a small pet dragon, who joined the family like a human member and was very close to Prinos.

The kingdom enjoyed a decade of peace, of almost unnatural happiness. In this time, the two children were born --- but on the day when Prinos turned seven years old, the days of happiness ended abruptly. They had neglected a group of people in the kingdom who had founded a new religion, praying towards a self-made goddess to grant them a more adventurous life. This religion had attracted more and more young people in the kingdom, who had become unhappy from the very happiness and peace the kingdom enjoyed. They begged for adventure, to become heroes and heroines fighting vile spirits, for a change in this stable clockwork of peace, which only translated into boredom for them. The king and queen had ignored this movement, assuming it was a youthful trend, a temporary strife to show off strength and power. But while it began as a few teenagers grouping together, it had become a religious movement after a year, and they had formed an idol out of gold, a female goddess sitting on a throne, graceful, thinking and reading --- it resembled Augustine quite well. She neither encouraged nor discouraged this movement, and while her parents begged her to publicly put a stop to this, she remained indifferent, which these people interpreted as further encouragement, projecting their feelings and dreams upon her. However, she was just uninterested in these religious movements and withdrew into her chambers to read and study, shrouding her in an air of mysticism, inspiring these \enquote{followers} even more. Her known love for magic stirred them into causing the upcoming incident.

When Prinos turned seven years old, these \enquote{adventurers} hijacked the festive event, causing an incident they must have prepared over weeks. And in addition, something happened which nobody had ever expected.

The birthday of Prinos was a public event, and also a holiday for all workers such that they could join the festivities. The prince himself joined the crowd, at the core of the festivities, on a square right on front of the castle --- it was a nation of peace, and while there were still guards around, nobody feared that anything would happen. And actually, it was not the prince who was targeted by those misled souls looking for adventure. They had constructed a mighty golem. The nation had experience with golems, since they used them for heavy physical work, but this was no work golem. It was equipped with a sword and a shield, and could activate some earth spells on its own. When the festivities were at their climax, they let it lose, and severed the link to its master, causing the golem to run wild in a chaotic frenzy. It approached the joyful crowd, and while they seemed indifferent at first, assuming a work golem had been misled, it caused a state of panic when it roared and started to throw balls of mud with its magic.

When the commotion started, Augustine left her chambers and gracefully moved to a balcony of the castle, watching the scenery unfold. At this point, the \enquote{adventurers} themselves dispatched. Essentially, their plan was not to harm anybody, but to show off their strength and present the thrill of a fight to inspire more followers to join their cause. But as you may expect, it all went horribly wrong.

The adventurers had mingled with the crowd, and now pulled their hoods back, exposing a red bandana around their heads which was the signature of their religion, red being the colour most prominent in Augustine's dresses. This stirred the crowd even more, tables were overturned, plates flew through the air, tents collapsed and buried the previously peaceful people beneath them. Before the golem even attacked in earnest, this panic had wounded many first casualties. Some slipped on the food which now covered the ground, others ran over them in a frenzy, and an elderly lady who had served the food before was the first victim who died in this mass panic. Cries of pain and shock echoed through the square in which they all had assembled before, and the fear intensified --- exposing a peaceful nation to a sudden attack was all the more effective.

Those who wore the red bandanas fell into a state of shock. This was completely unexpected to them, and some among them could not resist the fear which was spreading, and instead of fighting the golem, they used their shields and in some cases even their weapons to pave an escape path to flee from the square.

Prinos was quickly evacuated by the guards, who had not encountered real foes ever in their lives, but at least were trained for unexpected situations. However, nobody was there who could calm the people down, or even instruct the guards to attack the golem --- Ephraistos had been evacuated, too, and was kept locked in by his guards. Seraphina had withdrawn herself half an hour ago to prepare the birthday present for Prinos, and was still in a hidden room in one of the castle's towers when the events unfolded. And there was no general or supreme leader of the soldiers around, since it was a nation which had lost any reason to prepare for war.

The golem, infuriated by the loss of the magical connection to its master and becoming even more angry by the wild scene developing in front of it, went into a berserk mode, swinging its sword and attacking the people most close-by. The original plan of the \enquote{adventurers} was to parry these strikes, and fight back, but only a single one of them had made it through the crowd and stood in front of the golem, shield raised and knees shaking. This abomination of violence was completely unexpected to them, as is war to any fresh recruit --- they were left in shock and fear. It seems their fate was sealed when the sword of the golem hit the shield of the single bandana-clad soldier in front of him, and the wood was bursting from the heavy strike.

In parallel to this bloody event, some of these adventurous youngsters who had been derailed in their thinking enough to wish for any activity to happen to pull them out of their state of peace had assembled in their meeting place and prayed to their goddess --- some were dismounting the figure from the altar they had built, and the plan called for them to appear right after the situation was cleared, calling upon the crowd to join them in their religious fervour. They were already on their way when the crowd exploded into a giant panic, and people passed by their procession in fear. Unperturbed, or rather even encouraged, they continued on, approaching the square. When they reached it, the fight was underway and had already taken many casualties. Most of these adventurers were archers themselves, which is why they decided to take on the role of the ones escorting their goddess, since swords would be much more effective against the golem. Seeing the devastated square, some of them stopped in shock; others prayed to their goddess with intensified strength, and some decided to take out their bows and go on the offensive. It was them who attacked the golem with burning arrows, and who managed to lure it away from the square, below the water reservoir of their town, which was still close to the castle --- and with a coordinated effort, they \enquote{drowned} the golem. Naturally, golems did not breathe, but water was the natural enemy of the dry clay the golem was made of, and would dissolve its structure. Albeit there had been many casualties, the situation seemed resolved after this, but actually, this was only the spark to start the catastrophe.

At this point, the legend the fairies tell usually explains that human beings are strongly led by emotions, and logic or careful planning is often only an afterthought and impossible in the heat of the moment --- I believe you already know how convincing human emotions can be, and will grasp easily how the story unfolded.

As the situation seemed to calm down, most of the \enquote{adventurers} assembled in the square with the figure resembling Augustine. Now that the danger was gone and the fear had left their bodies, they still felt the adrenaline rush through their bodies, and started to pray in unison. All the lost lives were blocked from their minds, and they prayed to their goddess with renewed vigour, embracing this painful incident as the fulfilment of their dreams.

Ephraistos and Seraphina were led back to the square, surrounded by watchful guards, and looked at the scene with dubious eyes at first. Seeing their fallen friends and townspeople, they were filled by an incredible grief; and these worshipping people seemed to congratulate themselves on fulfilling their dream, neglecting the loss of lives of their comrades. The king and queen were at a loss when faced with this inexcusable behaviour which they could not understand; and when people do not understand each others actions and motivations, this is the seed to birth tragedy.

Ephraistos and Seraphina faced each other, and without speaking any words, they decided they needed to put an end to this movement. \enquote{Apprehend them!}, Ephraistos commanded their guards. \enquote{However, do not fight them with force, we can not afford to lose any more lives. Apprehend them in peace}, Seraphina added, and Ephraistos nodded understandingly. More soldiers from the castle were called in, and they slowly approached the group which was assembled around their goddess. The previously composed ruling couple had lost their serenity, and acted upon their boiling emotions.

As the religious worshippers were approached by soldiers, they formed an outer circle of swordmen, and some archers seemed to get ready, while their core still continued to pray to their goddess. More and more of their followers seemed to approach from the streets, some strongly motivated, and others just joining their friends for whatever personal reasons they may have had. The approaching soldiers caused an emotion of unjust treatment, motivating them even more than before that they were in the right, and those who were undecided and lost right after the painful event took place joined their comrades again. Emotional decisions always cause an emotional reaction in fellow human beings, and this reaction may backfire against the original intent.

The soldiers encircled them, but did not attack. They were at a loss on how to apprehend them in peace, when they stood in front of them, swords raised, with strong emotions and determination in their eyes, so the soldiers only raised their shields and it seemed a temporary stalemate was reached. Suddenly, a single arrow had come off, from the side of the \enquote{adventurers}, but it did not actually matter who started the upcoming fight; it sparked the fight among former comrades, friends, neighbours, and they had completely lost sight of their original intentions.

All this was watched by Augustine, who was standing on the balcony the whole time, while Prinos was still kept isolated within the castle together with the familiar of the royal family. Augustine watched in silence, with a graceful aura, and a completely unmoved facial expression. She did not know how to feel about these people, who seemed to worship a figure resembling her, who had killed many townspeople and were now fighting a bloody and intense war with the soldiers of the royal family. Was she supposed to act? And if so, how should she react to put a stop to this?

A part of her felt that this was the first time she actually saw something happen in this country, in reality, outside of her books. It was a painful bloodbath, and far from any of the adventures and dreams she had kept to herself when reading secluded in her chambers, but it was --- a ripple of change, in this never changing country. This was an emotion she could neither accept nor deny, and she could not even clearly say whether she felt positive or negative about it. In short, she was confused about her own emotions, and frozen in place on this balcony, taking in all the details of what happened on the square below. At this time, she was only six years old, but she had already grasped the concepts of life and death. She had been reading most of the two last years, escaping into the world of books, and in these fantastic worlds learned about many of the peaceful and warlike merits of human nature. However, this was the first time she saw people being killed, and there was nobody to take her away from that balcony, so she remained in place, completely frozen.

The clashing metal and flying, burning arrows below were completed by the background humming of the praying worshippers, who became more and more enthusiastic as time went on. Augustine was mesmerized by this view and sound. She wondered if a part of her own personality longed to be down there, in that figurine of the goddess, cheered on by the young fighters to bring this world more of this destructive change which clashed against the established, calm peace of the last decade.

It was this moment of thought, which was intensified by the natural cycle: Powers of creation and peace, clashing against powers of destruction and activity. Two sides of the same coin, split into two extremes --- it was this moment when the unexpected happened. A magical link was established between Augustine and the figurine which was worshipped by the people below, and the golden figurine came to life. The link was powerful and visible to all those in the square: A think, colourful thread which shot from Augustine's forehead to the golden figure below. The \enquote{goddess} raised, slowly stretching, and silently seemed to morph from her golden metal body into a human being. Augustine herself remained frozen in place, but something about her was changing: It was as if a part of her suddenly escaped her body, and migrated into the now more and more humanoid figurine.

A playful smile developed on the face of the \enquote{goddess}, as she came alive. She took in her surroundings, and emanated a bright light as she evolved into a human figure. Her worshippers were shortly stupefied, but then increased the intensity of their prayers, reflecting the smile which was shown to them. From there on out, it all happened in the blink of an eye: A wave of light originating from the \enquote{goddess} pushed away the soldiers attacking her followers, and it seemed to fill them with an unquenchable rage. Their eyes became red as if they were bloodshot all of a sudden, and while they had still been reluctant to attack their former comrades and townspeople before, they went into a berserk mode immediately. \enquote{I am Auguste}, the figure proclaimed, \enquote{and I will ensure that your boredom will fade for all times to come!}

With this proclamation, black wolves appeared, surrounding the worshippers and the soldiers. A giant fight evolved, and it was unclear who was friend or foe --- everybody attacked the closest being, and the clashing of swords upon swords, the bursting of shields and the crushing of bones filled the air. Auguste smiled, and laughed with full conviction. Meanwhile, Augustine's face changed to one of shock and fear. While she was previously frozen and unable to understand her own, conflicted feelings, all this was now gone and replaced by sheer, pure terror. She started to tremble, and shivering from head to toe, she started to run --- stumbling at first, almost tripping since her legs were shaking as they had never done before. She ran towards the safe place in the castle where her brother Prinos should be kept. On and on she ran, speeding up --- her step slowly grew more steadily, and she finally reached the room in which she would find protection. Then, before entering, she breathed in, and shortly closed her eyes, trying to compose herself; but the link to Auguste still existed, and as she closed her eyes, she could see what happened outside through her own eyes, and feel the rage and the empowering worshippers around her. She felt a sudden strength, and these feelings clashed with her fear, almost tearing her apart. Auguste was her, and she knew she was just an incarnation of a part of herself, empowered by these prayers and actions. Still, Augustine was unable to accept this, trying to shut out these feelings. It was her fault, for sure, her lack of self-control, which caused all this --- now, she was sure about this, and it struck her like a lightning bolt.

Opening her eyes again, she could see the door leading to the safe room where her brother was kept being opened by faithful guards, who had acknowledged her presence. She had also been followed by her maids, who stood by silently, catching their breath, but unable to act --- and then, she ran again, towards Prinos. \enquote{It's me, it is all my fault!} she cried out, tears streaming down her fair, tender cheeks. She ran towards him, and at the same time, depreciated herself; if she was at fault, she should not be granted this protection. She should pull back, resign to her fate, and die, to protect this nation. She closed her eyes again, as she saw the innocent look from her brother, who watched her without blame and with pure love. This was not right, she could not accept it. She was a monster, and she should be detained. How did this all happen, in the matter of minutes?

As Augustine thought all this, she took one step after the other, slowly tumbling towards her brother. She cursed Auguste, her fate, her indifference and her own tolerance --- how could she ever have accepted such vile feelings? She did not deserve to live, her memory needed to be destroyed. With these self-destructive thoughts in mind, she reached the arms of her brother --- and vanished into thin air.

The link to Auguste exploded with a loud bang, and Prinos was shocked. What had just happened? Confused, he looked around. There was no trace of Augustine left, and all he felt was a sense of sadness and tragedy clouding his heart and making it hard to breathe.

Meanwhile, the royal couple had started to evacuate. They had carriages prepared; even though this was a peaceful nation, a tiny bit of preparation in case of danger was still done, as if out of a long forlorn habit. The guards rushed them inside, and also Prinos and the family's familiar were led towards a secret escape path, and then directed into a carriage. The guards had to carry him at first, since he would not move after what had happened, crying out Augustine's name as if to bring her back. But there was no reply, and in the end, the whole royal family, barring Augustine, was moved to a remote town in their nation, hidden in a safe house only they themselves knew about. Even the guards did not know the exact location, since they had handed them over to other guards, until they arrived at their final location after several such handover procedures. The maids and servants who travelled with them had been blindfolded for the journey, and when they finally arrived, they required a whole day in silence to catch up on the recent events.

The main town and their castle, however, was taken over by Auguste and her followers. The townspeople had seen the furious soldiers and the black wolves, and were easily led to believe that this was part of a violent plan from the royal couple. While they had ruled the nation in peace for a decade, it is always easy to accept betrayal from those who might have been your friends, but never were really close to you --- humans easily accepted that others prioritized their own personal gain.

Auguste changed the nation quickly: She started using arenas previously used for competitive sport events for bloody, public fights, and began to breed dragons and wild creatures to let loose in nearby forests to attack unsuspecting villagers. They had undergone human violence to increase their thirst for human blood, and given these changes, it was easy for her to establish an adventurer's guild and give her followers the life they longed for. Nobody dared to ask the question where these monsters came from, and how they evolved. This was seen as a natural calamity, and they were quick to accept Auguste as their new, supreme ruler. Since she was still too young to lead most public events, she quickly accepted some advisers from her loyal followers who did her every bidding.

Auguste knew what had happened to Augustine, and it was also clear to her that the remaining royal family survived and would pose the greatest danger to her bloody rule. However, she did not know the location of the safe house, nor did she have another way to track them, since it was even unknown to the children of the royal couple. So she sent out some of her worshippers, clad in black robes carrying a sign of sword and shield, which had become her coat of arms. But it would take them a year to find the royal family, and carry out their bloody mission.

\froufrou{}

Aurora took a deep breath. She had communicated all this via her thoughts, at times also relaying vivid pictures from her own imagination. Apollo and Simetra sat there in silence, at times shivering at the lively pictures which entered their minds, and feeling as if they were a part of this powerful legend. And in fact, they were, but Aurora needed to conclude first to make them realize. Before starting upon the final scene, though, she took out the grapes again, as the sun had already moved high above them, and it was almost time for lunch. Apollo and Simetra gladly accepted, as these also quenched their thirst, and they also took a deep breath --- and finally, Aurora continued.

\froufrou{}

The whole nation was transformed, since the townspeople were now led to focus on their personal happiness, and not the happiness of the state. While they had previously performed many rituals and followed traditions together, their entertainment was now delivered to them in form of bloody arena fights. Adventures in the guild went out to find their own adventures,which left many out there to die, and other kinds of entertainment were taken by force, as slavery was not deemed a taboo anymore. The nation's wealth went downhill, and in less than a year it was already clear that to continue their new lifestyle, they would need outside resources. A war with the neighbouring nations was unavoidable, and it was what Auguste was planning for. She set up weapon factories and training camps for soldiers and sorcerers. Additionally, she recruited the winners of the most dangerous arena fights as her close, personal guard, convincing them to serve her with her magical and suggestive powers. Finally, she also researched the field of forbidden magic, extracting mana from the living, and channelling it into weaponry which would be used to attack her enemies.

A year passed by, and the royal family was still kept safe, but they knew that they would be inevitably be found. They prepared for this, and the former king and queen focused on giving Prinos a chance to live on and take over their dream of peace. To reach this goal, they studied shapeshifting and illusionary magic, which they carefully applied to Prinos such that he would not be recognizable by others anymore. They also hid his aura, sealing his magic and all his memories about how to use it, such that his characteristic mana would not be visible once he was found. Finally, they created a puppet which looked the same as he did before, and emanated mana similar to his, hoping this would fool the attackers.

And finally, on the same night when Prinos would turn eight years old, the attackers came. They knew it beforehand from some of their still faithful servants, but they did not flee --- the whole nation had become a land which was too unnatural to them, and most inhabitants had been made into their enemies. There was no place they could go. Their familiar, the pet dragon, had grown slightly, and took it upon her to hide the real Prinos from the attackers. They rushed in, and finished their task quickly, as the couple showed no resistance. Auguste watched the scene play out via a magical link, and was satisfied with the work. All of them had been killed, and no trace of the magic of the royal family was left. A magic symbol signifying the rule of the royal family was left in the castle, and it broke in the very same moment they died; it had been kept intact by their very magic, so August was sure they had been killed.

The attackers retreated shortly thereafter, ignoring the pet dragon which sat there in silence, almost like a gargoyle made of stone. The real Prinos was hidden behind him, kneeling inside a cupboard, and remained unharmed. The townspeople later on would bring him food and all necessities to survive for the next years, until he would mature at the age of eighteen. Such had been the agreement with the late royal couple, and nobody who was alive would know the real identity of the transformed Prinos anymore.

\froufrou{}

At this point, Apollo and Simetra interrupted Aurora, Apollo breathing in the air sharply as he was shocked at the resemblance of this story to his, and Simetra letting a silent shriek escape her beak. The legend had finally, and brutally so, connected to their own reality.

\enquote{The legend is not finished yet}, continued Aurora, \enquote{but I think at this point it might be best if you continue to tell me about your experiences from the last days, Apollo. Your mother has left you a message, and I am sure your sister has also contacted you in some way. Note that while the legend repeats itself every 500 years, it always concludes a bit differently, so while I can and will tell you the original story, the path you take will likely be different. But before that, let us have a real lunch up here, and digest the story over a meal.}

With this, Aurora produced a tiny, foldable table from her belt, which magically grew in size as she unfolded it and set it upon the wooden sign they were travelling on. She continued to put some tiny, wooden plates upon this table, which also grew in size. Apollo touched one of the plates, and while they were made out of wood, they seemed to be coated in a way which impregnated them against any dishes and maybe even soup. Next, Aurora took a small chalice from her belt, and poured something into the plates: Indeed, it seemed to be soup. The plates were deep enough that she could fill them quite a bit, and even Simetra would be able to eat with her beak. For Apollo and herself, she added two wooden spoons to complete the lunch table. She clasped her hands together, and said a small prayer to thank nature for this meal --- and then invited both of them to dig in. Silently, they gladly did, all lost in their individual thoughts.
