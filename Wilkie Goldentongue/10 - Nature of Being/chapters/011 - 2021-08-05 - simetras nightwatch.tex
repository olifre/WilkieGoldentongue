\chapter{Simetra's Nightwatch}
\label{cha:simetras-nightwatch}
A strangely lifeless, but relaxing silence encompassed them. Simetra was keeping watch, as she had promised and could only hear the air rushing by as they continued to fly over an endless sea of trees. This was certainly no ordinary forest, and the more she watched the trees, the more similar they appeared to be to each other. Simetra rotated her head --- she was an owl, and while she was not the regular kind you may meet in an ordinary forest, she had inherited many owl-like features. With a keen eye, even though she was half-asleep, she could rotate her head and watch their surroundings. They must have been flying for hours by now, but still, in all directions, there were only trees to be seen.

There were so many questions on her mind that it was hard to stay half-asleep. What was this huge forest about, which was constantly being destroyed and rebuilt? How did they manage to find the fairy in this vast sea of trees, and why was she wandering through these lands? And, most importantly: What was going on with the world outside the village they had left?
The village had been mostly self-sustainable, but still, traders and travellers passed by every now and then. Where did they come from, as they were surrounded by all these trees, and how could they pass through these lands with their goods?

Simetra decided to stop fretting about these questions. She needed to make use of the calm hours. After not even two days had passed, more unexpected and completely stupefying experiences had rained upon them than ever in their lives. They all needed some time to process what happened, and to even understand which questions might be most important to find answers to first.

Relaxing a bit, Simetra focused on herself. She had worked magic today --- unthinkable just yesterday, and still, while it felt so natural to her in the moment, it now felt like a dream. A dream constantly proving to her that it was real, by presenting its results right in front of her attentive eye, protecting them from a cold and biting wind.

Magic seemed to be so natural to her, as if it was engraved into her soul. It had taken a toll, and drained her energy, sure, but she had always imagined there were books of magic and old tomes which contained this wisdom. The magicians which had passed through their village made it seem as if this was the case: They had been carrying books with them, watching their audience with the eyes of a sage, when they showed some small magic tricks to gain some money from the crowd. There was nothing like the feeling of being connected to nature and to themselves in their eyes --- this was very different from what she had experienced now. Also, they moved their hands quite differently, in slow, excessive movements, as if the motion was rehearsed over and over again.

Simetra recalled these memories from their past, and decided that when their relaxing slumber was over, after catching up on exchanging more than just their names, she'd first ask the fairy about the nature of magic. After all, Aurora's magic was what caused Simetra to gain a voice, to be able to express her thoughts in human speech. Understanding this marvellous miracle was most important to her.

It was the second time now that Simetra switched over from one eye to the other. Several hours had passed, and the sun was glowing in a dark orange, touching the horizon. The trees were dyed in a less deadly, warm colour, but it was only the sunlight granting them a more vivid appearance. It would not be long before night would fall, given this flat horizon --- Simetra had rather keen night vision, but that would not protect her from the unnerving feeling that those dark, lifeless surroundings would become mostly invisible to her. They barely reflected the sunlight, and would be even more impregnable when bathed in the dim light of the stars, hiding any danger which might be lurking within from her sight.

As she was slowly accepting this lonely, unnerving night to come, the sun finally vanished behind the horizon. The last specks of light slowly faded away, marking the end of dusk, and the stars shonw brighter and brighter above them, led by a waning moon slowly fading into her view. After taking in the nostalgic night sky, she forced herself to look down onto the dark forest --- and she had to squint. The forest was far from the black, impregnable and silent mass she had expected. Somehow, there was a strange, cold light emanated by it --- no, \emph{light} was the wrong word to described this view. It was as if there were tiny fireflies rising from the leaves, dancing in an orchestrated pattern, miniature enough that it could also be some kind of dust, emanating a white and cold light. She had not seen these bright particles in their last night in the forest, and they somehow reminded her of the magic dust Aurora had with her, but in a less natural, more mechanic, somehow cold colour. The particles were slowly rising, then following an invisible stream or flow of the wind along the treetops.

\enquote{Harvesting}, was the word which apparently jumped from nothingness into Simetra's mind. Did she finally witness the purpose of this strange, seemingly dead forest? She tried to make out the direction these particles were taking, orchestrated by some unknown force. However, the more she focused, the less clear the direction became --- whichever direction she faced, the glowing dust from the trees which were close-by seemed to move away from her, but lifting her gaze a bit higher, it seemed to come closer, as if to confuse her about the actual direction it was moving in. Still, without being able to name it, she was sure there was a deterministic movement, as if some kind of harvesting or collecting was going on.

With some effort, she averted her gaze from this mesmerizing, somehow beautiful sight. This must be magic at work --- it would explain why the dust looked so similar to the magic dust she had seen on Aurora, albeit it was bereft of any colour in this case. Someone, or something, was harvesting magic from this dead forest, this is what she felt instinctively. But what was the origin of this magic? How did the dead wood produce it?

Following her instinct, she woke herself up, stretched her wings, and as expected, some fluff fell down from her feathers. Carefully beating her wings to produce a bit of wind, thoughtful not to wake any of her sleeping companions, she made the fluff fly away, leaving their wooden plate, approaching their shield of feathers. She followed it with both of her now fully alert eyes, and imagined the fluff to pass through their protective barrier without being hindered by it. The protective sphere seemed to comply, and while the fluff was quickly moving farther and farther away now that it was out of their protective sphere, her eyes which were well-suited for nightly observation zoomed in on this small piece of herself. Slowly, it tumbled down, approaching the treetops. When it must have touched the first of the dead leaves, the light began to change --- as if trying to greet the piece of fluff, they moved closer first, concentrating around it. Then, when the fluff must have passed through the treetops, a bright light erupted, and many of the white dust particles suddenly exploded from the trees as if a bright fountain of light had appeared. Focusing closely, Simetra could also make out some colours in the glowing fountain. All this only took a few seconds, and a ripple went through all the particles around that region, as if this was a large ocean of dust. Then, the bright fountain died down, and the scenery changed back to the calm, but unnervingly constant and repetitive stream of dust all around her.

\enquote{As if the forest had eaten the piece of fluff, or some component of it}, Simetra silently thought to herself. \enquote{Did it draw out and absorb the magic within? Was this the reason why some colours appeared in this fountain?}, she pondered. \enquote{Was all this huge forest a single field to harvest --- magic?}

A shiver ran down her spine. These trees, which looked dead, felt as if they were against nature, absorbing energy and directing it somewhere else. With her limited knowledge about magic, Simetra still knew that magic was present in all living creatures. Clearly, not every being could channel it, but ancient stories she had read with her friend clearly said that magic was part of the force of life in every living thing, and remnants of magic would spread to any inanimate objects living beings touched and worked with, like memories are shared with other living creatures.

This meant that likely, this forest would be a deadly trap to any living creature. This could be the reason why not a single animal was to be heard in the forest, and there was a complete, deadly silence below the treetops. They must have been rather lucky to encounter the fairy, and escape from this trap, before their energy was drawn out of them. This might be part of the reason why they were so utterly tired, and it might also have been the cause of these strange happenings in the past days. This fountain of light, caused by simple fluff, and these ripples throughout the flow of magic dust must have been very visible also for their every movement through the forest. It was strange that all this seemed completely invisible from below, even at night. Or maybe it was to be expected, for a forest luring in victims to extract their magic energy? All this only added to the questions on her mind --- even though she had come to conclusions, the \emph{who} and \emph{why} was completely in the dark. And, naturally, it was completely unclear how this large forest was \enquote{constructed} and how the strange maintenance cycle worked. But another question was more important for them right now: how large this actually was, and when they would reach the outskirts of this gigantic field.

With all this in mind, Simetra knew she could not come to new answers on her own. While her mind was still full of questions, this gave her some kind of closure --- enough to first shoot a quick glance at both her companions who were still sleeping safe and sound in this ball of feathers, and finally then to close one of her eyes again, entering the half-asleep state she'd keep up during the remaining nightwatch.
