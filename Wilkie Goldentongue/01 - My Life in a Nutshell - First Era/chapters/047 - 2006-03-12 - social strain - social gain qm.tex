\chapter{Social Strain --- Social Gain?}
\label{cha:social-strain-social-gain}
\subsection*{Originally published: \DTMDate{2006-03-12}}
\begin{quote}
Hello, once more!

Though it's weekend, not time is left to write something very long. The exams are just closing in, and I'm bound too learn for them --- so many things I'd like to do, and no time is left\dots{}
However, I still managed to do this short text here. Please enjoy it!

Is this a contrast or a similarity: Social Strain --- Social Gain?
\end{quote}

Weekend, once more. And several things had happened. Before we start to deal with these, we shall go into detail about some event that had happened on Friday, on his way home; I promised you to tell you about it, and I guess it will be pretty interesting.

He was just on the way to the bus, and he knew he'd have to hurry if he wouldn't decide to take the train. When he was just around the corner, some 200 metres away from the bus stop, the bus would arrive. Then, he'd decide to take the train, and stop running. However, the bus didn't leave immediately, but the doors would stay open for some minute, and he guessed that a passenger was buying a ticket, though he didn't really believe in that. Some seconds later, he decided to start running again, and when he was just in front of the bus, about two or three metres away from the door, it closed, just to open again a second later. He entered the bus, and the busdriver told him he should have run the complete way. Our protagonist explained that he'd already started running long before and could not have gone faster, though he'd just not expected any busdriver to realize such details as a student running to catch the bus. This one had, apparently, and my belief in the simplicity of the senseless and careless world was shaken to the ground.

He'd seen a girl inside the bus from the outside, and he was pretty sure that this was the same girl he knew, the one to whom he'd talked when trying to catch that bus O. had taken; the day when the train had been very late and he'd missed exactly that bus here. At that time, he'd chosen another bus, and this girl had been talking to him; and when he arrived home, he'd phoned O., for the first and the last time up to today.

Had this also happened on a Friday? He was pretty sure it had, as his parents had not been at home; they always arrived later on Fridays. 
Otherwise, he'd not have phoned her, of course. 
His love was to be a secret, even to himself.

But we don't want to go back to the past, though this moment, he'd realized now, was quite similar to that moment in the past; the difference was, however, that O. was \emph{not} on her way home, though she was finished with her lectures for that day, too; and she hadn't talked to him about the bus the two of them would decide to fetch, as she would probably stay some time longer to do some work.

Cars were a sign of power; destructive power. At least, when it came to emotions; and sometimes, they could even cast of physical destruction. Everything had two sides, and we already know that these sum up to --- zero.

For that reason, nobody and nothing could be perfect, except nothing itself, and as long as something existed, nothing would be perfect. But the sum of everything out there would be perfect, too. 
Sometimes, he assumed that there was everything everywhere, and just chance would decide which kind of matter would exist or be seen by us; an interesting theory. 

But now, we want to stick to the present.
He didn't sit down next to that girl; the last time, he'd done, because she had symbolized him to do so by taking away her back. This time, she couldn't, as she already kept it to herself, and if she'd done so before he'd seen that, he hadn't noticed. The seat in front of her was not such a nice place to be seated, and thus, he decided to sit down some metre away to her left.

The two of them began talking, and when the humming of the bus became so loud he couldn't understand her properly, he decided to seat himself in front of her, finally, turning his head though he knew his neck would become stiff. He ignored such things, as social contact seemed more important to him at that time. He was shocked by the fact that she didn't have no problems with looking into his eyes, and he didn't flinch, but return her friendly stare; as a friendly gesture, she offered him a sweet, and he accepted, casting off a discussion about sweets; the result was that the two of them were fancying the same things. Interestingly, she told him indirectly that she was having a boyfriend, but she put some accent on that word he could make out; it told him that some part of her was insisting on passing on that information, as if to cast off some reaction or as a kind of protection or defence. It seemed to be an intrinsic part of human nature to pass on such things; but he hadn't known when he was together with O., and thus, he didn't notice a thing\dots{}

His contact with her had shrunk more and more, again, and he was not so keen to see her on Monday, as she would probably continue that game of friendship, that simulation without real foundation; he wouldn't know if that meant that she was trying to forget that she'd once loved him, or if that was to say she'd found him to be boring and just kept up some kind of friendship so as not to offend him.
Probably, he didn't want to know.

There was something his best friend had told him when talking about his love to R.; at that moment, when he'd told our protagonist about that the first time, he'd advised him not to tell anybody; at the same time, he'd figured out three options: Either, he would tell her and she'd love him; or he would tell her and she'd not love him. And the third possibility was that he didn't tell anybody. This would finally mean that there were two positive possibilities, but one of them wasn't really probable. One thing we all should know is, that there are always two possibilities to everything --- \emph{at least} two possibilities.

R. --- he'd once communicated some time with her, digitally. Then, he assumed that she didn't want to talk to him anymore, as she'd suddenly stopped contact and didn't answer him anymore. In real life, she'd only talked to him once; then, she seemed friendly, and he couldn't remember having offended her. Probably, she was using some kind of profiling system, too; Of course she was, everybody did so, somehow, sometimes.

B. --- she'd contacted him on Saturday evening, and told him that she'd blocked that thing he'd shown to somebody else, now telling him that he should not have done so. He answered her in a whole bunch of words; when he was finished, he felt he'd written too much, but couldn't take the words back; words can't be controlled anymore, once they're left on their own. She'd once been offended when he'd talked to her a lot, guessing she'd like to read something; it seemed she hadn't. He felt offended, as he'd thought she was interested in literature; it seemed she wasn't. 
That's one of the reasons why he thought that he overestimated her --- sometimes, he'd think the other way round.

P.'s friend had also talked to him, and he'd enjoyed that communication. She was never offended, and when he said goodbye, she was gone, just having stayed to talk to him. 
We don't know what is yet to come, what is lying ahead, behind those clouds blocking the future, and we're not to find out until they have arrived. 
Please be patient, it will probably take me some time to find more time. 
Meanwhile, you can make up your minds about some opinions\dots{}

\begin{quote}
Time is flowing. \\
Time is showing \\
what has happened. \\
Time is killing. \\
Time isn't willing \\
to give in. \\
I'm not to give in, either; \\
but one of us has to do so. \\
At some moment \\
of time. \\
--- W.G.
\end{quote}

\begin{quote}
Emotions \\
control ourselves. \\
We \\
control our planet. \\
The storm \\
controls the living. \\
Gravity \\
controls the stars. \\
Time \\
controls our universe. \\
And chaos \\
reigns everything. \\
--- W.G.
\end{quote}
