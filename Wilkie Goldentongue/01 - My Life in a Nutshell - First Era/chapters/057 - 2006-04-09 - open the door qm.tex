\chapter{Open the Door?}
\label{cha:open-the-door}
\subsection*{Originally published: \DTMDate{2006-04-09}}
\begin{quote}
Hello! It's me once more!

Before we start, I want to thank all those people out there who support me by programming this blogging system and by reading my stories. The new system is wonderful, and I look forward to all the new functions that are still to be added. Thanks, OliFre, and all my readers out there!
But now, I shall continure with the story. As I've already told you, it seemed probable that I would fall silent for some time; However, things were different, but this is the normal way of development. Many new pieces of information have arrived, and some old notes from last week have been found; you'll see what I'm telling you if you continue reading.

Should he: Open the Door?
\end{quote}

\begin{quote}
Taking a chance at something new in the near future will pay off.\\
--- from a small fortune cookie program
\end{quote}
This quote was the one that had influenced him after the things that had happened the last evening. At the moment, he could only contact people by means of digital communication, and he'd done so yesterday evening.

But before we go into detail, we should deal with the old notes, as there had already been some influence some days ago.
A look at the calendar had revealed that it was the name day for a name which matched O.'s nickname; he'd never believed that this name would really exist. But it gave him a shock when he realized that this meant he'd have to look it up in order to find out about the real meaning of this nickname; though he knew that it was a name of a species, he realized that it would have to be the name of a holy man or woman to be used as a name day.
He also knew that she wouldn't care about it, as she was protestant; but he did, as he wasn't.

He decided to look it up as soon as his computer was ready once more, as it kept crashing; O's computer did so, too, an thus, there was no way of communicating with her via digital media --- he'd tried, but she seemed not to care.

But the main thing that was of importance to him now was that he'd communicated with R., and for quite a long time; the same girl his best friend loved without telling her, and the same girl another friend of his was in love with, too. We've already realized that this emotion was based on her smooth skin, which was apparent to everybody who wished to notice; he felt the impact it had on everybody who watched her, and he only did so if necessary. And he knew that it would be very simple to fall in love with her, if he wished to do so.

This fact made him realize that love could be controlled by logic if you knew the way it worked; However, fate could still interfere. Several signs arrived spontaneously.

First, R. had sent him a picture; it was two or three years old, at least, but it showed her in a childish way, which is one of the most attractive way to present oneself. Of course, as a good friend, he'd also sent this one to his best friend; but then, when it was too late, he realized that this would not help him, but make his desperate and senseless longing only stronger.

He remembered the day, when R.'s brother had tried to search a girlfriend for him in another childish manner; he had told him a lot, not saying a thing, leaving him without any real information. Nobody could penetrate the shell that was not to be broken, only one person could, and he was still waiting for her. This person had to try hard, but she'd be happy when she succeeded\dots{}

This was something that applied to all people who were interested in technology and mathematics; they were bashful, but alo truthful if the shell was broken. However, this would mean that all their feelings would be concentrated in this shell, and that they had to experience a lot of pain; and he was very happy that he could also express his feelings by writing down all the things he felt. 
He realized the weight of the decision to give his innermost parts away; but it would not mean that they were lost, but that they were shared and become even more wonderful. Even if nobody would ever read these lines, they had helped him a lot, and he knew that he could not lose a thing but time he'd have spent contemplating.

And writing \textbf{was} contemplating.

He was sorry that he could not really wait for R. to answer to his saying goodbye the last day, as his computer had just crashed once more; on the other hand, they had talked a lot, and though it was probably no real friendship, it was comradeship. He wondered why he was always a comrade to all the girls and boys around him, and why he could talk to all of them without having a girlfriend. 
Probably, it was just a matter of time and he'd have to wait.

He'll do so, and continue to learn about his fellow creatures and himself --- which is the same, in the end. 
The next report will probably take it's time, but you should know that one can never be sure of that.

To conclude this one now and here, we will finish it by looking up O.'s nickname in the dictionary; and except from that special species of birds, he didn't find a thing. 
But he'd realized that his mind \emph{had} to be on somebody; probably, he should continue focussing on people that would never return his love to stay the way he was. But he didn't know; time will show, hopefully. 
Thus, we shall finish now; please stay with me, and tell me what you feel about it.

\begin{quote}
Free \\
like a bird in the sky, \\
like a fish in the water, \\
like a slave \\
on his way home. \\
Captured \\
like a bird in a cage, \\
like a fish in the glass, \\
like a slave in the field. \\
That's what he was, \\
and all of it \\
at the same time. \\
--- W.G.
\end{quote}

\begin{quote}
Pain and happiness\\
are relative. \\
From the absolute point of view, \\
they are both completely \\
equal. \\
And in the end, they sum up to nothing, \\
leaving us hanging in the middle \\
of emotions. \\
--- W.G.
\end{quote}
