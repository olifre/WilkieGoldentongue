\chapter{What the heck is going on?}
\label{cha:what-heck-going}
\subsection*{Originally published: \DTMDate{2006-02-09}}
\begin{quote}
I'm back for you again!

A lot of things have happened, maybe even more than before\dots{}
However, he didn't have a lot of time to write about it; Now, we'll try to catch up!

My question is: What the heck is going on?
\end{quote}

Wednesday, Thursday --- days were passing by in some sort of terrible rush. 
And he was failing to catch on. 
We'll try to look through all of it chronologically: We'll start with the things he forgot to mention on Tuesday. 
And probably, he realized them just after he'd finished writing. 
When he saw O. passing by that day, he noticed something special: Not only was she walking in the leisure way she always used; In addition to that, she walked \emph{very near} to doors, something he'd also do quite often. 
At least, until one of the doors had opened and smashed his nose.

After that, he tried not to go near to those doors again; However, it seemed to be something that was borne, unchangeable in his mind. 
Maybe, in quite the same way he felt that he had several features in common with O., though he knew all the same that they were completely different, somehow. 
The perfect mixture\dots{}

It was gone. \emph{She} was gone. It was all over. 
And maybe, this thing with the doors was just an effect of her always walking to the left side of her best friend, while he tended to walk to the right side of his. 
Peculiar thing.
But that was something that happened on Tuesday, and now, we're going to have a look at Wednesday. 

Wednesday --- a day full of work. A boring day; He was drowsy, and he was even near to sleeping in one lecture. But he managed not to fall asleep. 
That was pretty hard, but he knew he could do that by simply talking to somebody, and he did so. 
Nobody seemed to notice that, however, as he was able to hide the drowsiness with his profiling system --- which was burnt in his brain, in quite the same way other persons move their arms without thinking. Thus, he could do so really effortless. 
And he did, all the time, as there was nobody who was so important to him that he would turn it off. He was pretty sure he could, and he'd once tried to do so when he was together with O.; And the wisdom that there was the power of an own opinion residing in his head gave him the energy to believe in a better future, which meant that there wouldn't be such a thing. 
But he knew that, too, which left the future being something unpredictable, something rather random. 
After typing these sentences, he tried to enter the last four numbers of O.'s password; He knew it, and he still remembered everything about her. 
She'd used the numblock, and he'd never used it to enter the last four numbers of his password; Something else they had in common, that both passwords ended with four numbers; On the other hand, they'd used different parts of the keyboard to enter them.

Hers would be hidden in a better way, and thus, he tried to imitate her. 
To the right of him, O. was passing suddenly, though she was supposed to be somewhere else. 
She didn't notice him, though he was typing quite fast, and though all computers were supposed not to work, as one of the students had killed them with a virus. But he knew how he could operate them........
No, he hadn't been that student, but he'd have had the abilities and the knowledge to do so. 
O. was searching for somebody, but of course, she didn't ask him. She was doing that on purpose, he felt, or maybe, fate was controlling her --- or that mystical conscience that nobody knew about, but which was somewhere out there, and maybe inside all of us, too.

O. was asked whether she wasn't at the place where she was normally supposed to be, and she simply answered: "No, I'm not supposed to be there."
And then she stated she was in search of some other boy, which was probably needed at the place where she should have been; that would explain the slight irony in her voice, something he was now just able to notice. 
But that sort of irony was so normal with her, that nobody noticed it, really; Thus, they finally were offended and fell silent. He didn't. He cast a look around, seeing P., sitting several metres away, talking to some girl sitting next to her; She showed her some book she'd shown him before, the book she was now about to start reading. It proved his knowledge, that she liked talking to everybody about everything, once more, and he felt that she only talked to him as much as she did because he hadn't offended her. 
Most people did.

P. seemed to be really bored now, however. And she didn't think of disturbing his writing. Was it some kind of \emph{tact} or something else?
He wouldn't know.

Finally, we're bound to continue with Wednesday, if we really want to arrive at the present in the near future. 
He'd seen O. twice that day, once together with B.-B., and once alone. 
But there was something more important developing. 
Wednesday morning, he'd talked to Y. about P.'s friend. 
She'd seen that girl dancing around him in the library, and she was wondering whether there was something \emph{under development}. Nevertheless, she'd also noticed that he didn't make it easy for that girl, if she really wished to \emph{achieve} something, as he just thought it to be some kind of fast developing friendship. 
Maybe, this was the same to her, as her character was in some kind comparable to P.'s.

Another girl, which was also one of P.'s friends, sat next to him for a complete hour, just playing a game on the computer --- In reality, he wished to start writing, but the way the two of them were talking implied some kind of friendship though he didn't really kappen to know her. Finally, this told him that her character was also comparable to P.'s, and as they were friends\dots{}

Thinking about friends: That girl that once stood on the bus stop alone, just having contact to him and P., was now having a new boyfriend; Change was everywhere\dots{}

She was listening to a song, and he knew the original version of it, and even the names of the persons who'd done the composing; When he'd told her that he knew that song, she'd simply given him her headphones for some seconds. He'd realized immediately that this was another version of the sond, though she believed it was some sort of original; However, she also knew about hundreds of different versions. 
In such a way, most people were talking to him, and he wondered whether it was real interest or just smalltalk. O. had done more, when she'd been single\dots{}

Today, P.'s friend was talking to him, but she liked fussing around, mocking him for fun. He did the same, as it was part of his profiling system. 
A terrible anger cought hold of him, when he realized that this was her character. How could such a childish behaviour be normal? She was intelligent, of course, and she could concentrate. 
Most of the people he liked were that way. Thus, this vivid, lively way of acting was also part of O.'s character; Maybe this was one of the reasons for his anger. Nevertheless, he knew that she was somehow different: She could turn over completely, and he'd noticed that she was capable of thinking really complex \emph{without} laughing. In addition to that, she had some sort of profiling system, too, though he realized just now, when looking through his memories, that her's had been deactivated when she'd been together with him, and without any boyfriend.

Truth is the basis of every relationship, everybody knows about that; Thus, such a profiling system \textbf{must} be deactivated if one searches for something \emph{real}. The problem with this is, that the system simplifies all those relationships. 
But love's not simple, at least not real love. 
And we shan't think in such manner of it. 
Another thing he'd realized was that O.'s former boyfriend had left her, and not the other way round; A consequence was that she was alone, crying, searching for the help of her mother\dots{}
The fact that she'd told him about this explained that she trusted him, even though she had a boyfriend at that time --- her current one. But it also implied something else: She didn't love him. 
That occured to him once again.
And maybe, she'd never done, and it was only some misunderstanding of her character; But he wasn't alone with this problem, as we'll see when going on reading.

For Y. told him one morning \emph{(the very same morning she'd talked to him about P.'s friend)} that it would be quite hard to take a chance on him. She described the way that girl tried to talk to him; She meant the moment when he was typing, probably. But Y. didn't know about that texts. 
P.'s friend did, as she didn't have no connection at all to O.

Or to Y. And, in fact, though she was interested in the story and had read the beginning of the first posts, she didn't want to continue. However, she'd asked him if O. was real, and he'd denied that; Thus, this could've been the third denial. If P.'s friend was really aiming at something, then, it appeared to him, that her behaviour was childish or even foolish. 
And in the bible, there had been three of them, before the cock croaked; Was this another sign?
He'd just realized that fact, and he didn't know what it would mean\dots{}

But we're still not finished with Wednesday: The number two seemed to be of importance that day, as most of the numbers he read could finally be summed to a number, whose checksum was two. Most crazily, even when the temperature outside and inside would change, the checksum would stay the same. 
Something else he couldn't really make something of. 
Probably, it was the expression of his thoughts and memories that raced through his consciousness; Maybe, this was the way they influenced matter. 
On Tuesday, it had been decided that he was to work together with P. on some sort of project; the rest of the group wouldn't do a thing, he knew that, and she did so, too. 
Thus, the two of them would take over all of the tasks there were. 
On Monday, this work would be finished, and 'till then, we'll have a look at their progress. Up to now, there was not much they'd done, but tomorrow, they are going to meet to finish the project, or at least, to try to do so.

Something else: Today is Thursday. 
And we'll now have a look at the things he encountered today. The contacting attempts were still without success, though O.'s best friend had smiled and waved at him about one day ago, sitting leisurely next to another boy. He was pretty sure she had a boyfriend, but nevertheless, she let him touch her belt. 
Sometimes, he thought life was a game, and some people seemed to do so all the time.

But today, more things happened: Further contacts with P.'s friends \emph{(this time, there were more of them, and they'd all been girls)} were to come. 
And the way they behaved told him, that the things P.'s friend \emph{(the one we're still searching a name for)} did were normal for this group of people. 
Absorbed in some kind of childish game, and even if they weren't absorbed at all, they'd talk to you as if they'd known you for years. Somewhere, he'd read something about it; Ahh, yes, he remembered. There was some discussion about the way American, British and German girls thought about kissing: For some of them, it would be just a sign of friendship, for girls coming from another country, it would be the sign of a strong relationship full of love. 
People were different everywhere.

Something else happened today: Another boy asked him, whether there was some \emph{real} relationship between him and P.'s friend, and he'd denied it, as he didn't even know himself. This guy and some of his friends mocked him, but finally, he managed to laugh with these people about himself. Thus, such a relationship would look improbable. 
Another guy had asked him some days before, one of his friends, and he'd denied it another time; The only person he'd really told the truth, as she seemed to be interested, was Y. 
In fact, she mocked him, too, but in a way he liked, and she knew that.

But Thursday is still not finished; He'd talked to a professor about some project he was going to do, in the same hour he'd once proposed to O. and her best friend as the hour to meet --- weekly. And as they'd never ever talked about that again \emph{(probably, her friend hadn't even told her about it and forgot about that on her own account)}, he didn't feel any remorse or guilt, at least not \emph{really}. 
Just before he'd go to the bus stop, P.'s friend was playing some childish game with him again, just after he'd said goodbye: He didn't understand what she'd said, but finally, it would mean he wasn't allowed to talk up to the moment she said his name.

She really seemed to enjoy this stupid game, and he didn't say a word. Then, she would be hurt, probably, but she seemed to be enjoying it \emph{(or at least pretended to do so, so as not to give him the possibility to make fun of her)}. 
This fact is describing her quite well. 
At least, the part with the childish game.

Nevertheless, he'd also noticed that she was pretty intelligent, only didn't he notice that in her way of talking: No puns, no quotations, nothing at all. 
Probably, her intelligence would only be mathematical?
Well, that would be quite peculiar; His theory induced the thought that intelligence was just a matter of interest, as we've learned already. Should he help her?
The question is yet to be answered --- later.

For now, this one is long enough, though I notice that the less I write, the more there is to write about. 
Please stay tuned, and look forward to the next post!

\begin{quote}
Being puzzled \\
is the only way, \\
to give future the chance, \\
not to be the opposite of one's thoughts. \\
--- W.G.
\end{quote}

\begin{quote}
Silence --- \\
it was night, and silence was all around. \\
Only the whispers of his own thoughts were to be heard. \\
And the powerful wish \\
for unity. \\
This wish \\
escaped his mind, \\
and went outside \\
to search for \textbf{HER}. \\
--- W.G.
\end{quote}
