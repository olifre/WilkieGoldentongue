\chapter{Confused}
\label{cha:confused}
\subsection*{Originally published: \DTMDate{2006-03-02}}
\begin{quote}
It's me, again. 

Hope you're still there. This time, we'll just have a look at all those tiny details that had been forgotten the last time; and the confusing facts that had just arrived.

All that left him: Confused.
\end{quote}

All those new sensations were chasing him, and he wouldn't know how to react. Thus, he was left confused; but before this condition was to become reality, we shall have a look at all those things that have already happened. Yesterday, for instance, when he'd had a look at that image that seemed to resemble some part of O.'s character though it looked completely different, he'd done something just before the lecture began: A girl asked for somebody to close the door, as it was getting cold, and nobody expected anybody to do so; he did, and some of his friends seemed shocked. He'd always felt that this girl was something special, having some things in common with B., and the colour that was emanating from her aura seemed quite similar; she liked reading, and philosophical discussions. He'd felt that she would be not as shallow as many of the others were.

And he closed that door without thinking about it, and finally, when they all entered the auditorium, he'd sit next to her without intending to do so, as he'd just chosen a place from which he could perfectly see what was going on, and at the same time, a seat close to the place he'd chosen the last time.

Thus, one might say, the two of them were led together; however, he felt that this deed would probably insult her in some way. Finally, she seemed to stay quiet, mostly talking to other people in that lecture, while he wasn't sure that she'd really thought in such a manner. He wouldn't know, and he felt that he would probably be taken as flirtatious now, as that evening, he was talking to somebody who assumed him having a relationship with P.'s friend, while he denied it by declaring this connection as a strong friendship. But he wasn't so sure about it, as some things would happen this Tuesday which confused him even more.

Tuesday --- today was the day of his not knowing what was really going on. He'd like to be able to read the thoughts of some people, but he couldn't, and was feeling that his profiling system couldn't satisfy all his needs. However, it was the single system that had been the foundation of his life till today, and he wasn't sure if that could be changed.

Everything could be changed. Nothing was borne, with the possible exception of change itself, and nobody could really deny this fact or prove it to be wrong. L.-B. had been vanished for this day, and he'd been glad to be alone with P. and her friends; and his other friends, of course. The phone rang several times that evening, and he was happy to be able to talk to living persons, though he couldn't really solve all their problems.

Something pretty interesting had been happening this morning, at university: Several things, in fact. He'd been together with P., her friend and the others for a pretty long time, for half an hour beging alone with P.'s friend who seemed to be really interested in his personality. He felt confused, when P. told him indirectly, that she didn't like the colour \enquote{pink}; well, he'd always seen that colour in her clothes, though it had been hidden quite well among other colours, and now, he was assuming that the colour that one happens to emanate is not necessarily the colour one \emph{likes}; probably, it's \textbf{never} the favourite colour that we can see on the surface, but the exact opposite. Thus, he'd understand something quite complex: The things we most hate are part of ourselves, and publicly available in the way we deal with others. As a conclusion, the things that are happening on the surface are always the exact opposite of our real characters, and we can't really judge people when having a first look at them; however, our subconsciousness seems to have realized that fact, thus making those \enquote{two-minute-decisions} that we do in the blink of an eye most times more trusting and fitting than decisions that may take us a long time, because both are arranged \emph{in contrast} to each other. Which would completely change our view upon the world; as now, we wouldn't be able to judge persons in regard of how they behaved or what they were doing. This was logical, but logic could \emph{never} be regarded as the foundation of the world, as it predicted itself that the world would grow more and more chaotic every minute.

But we want to continue with the present, or, at least, the steps which had led us there. In the morning, Y. was there again; but this was the first day since long, when he hadn't reserved a seat for her, as he didn't like to talk to that woman again --- which hadn't arrived at all that day --- and as he didn't expect her to come, as she'd left him alone for weeks.

Alone in the bus, that is, as he'd explain her and her friend some things that very day; this had been announced yesterday, as you may remember. He was happy of being able to help those two girls, who had never gone on his nerves, while he'd stay with them even some minutes after they were finished, as he was pretty sure that there was nowhere else to go, and he didn't want to start working. Those two, however, seemed to be not doing anything, and finally, they even managed to bore him, the person that was never bored by anything. At least, when he found some other place to go, he did so, saying goodbye to the two of them and finally being alone with a technical device once more, until P.'s friend joined him, soon followed by P. herself. Fooling around together, the time till the next lecture would begin passed quickly; then, P. would tell him something he hadn't expected. Firstly, she'd indirectly told him she didn't like the colour of pink; secondly, when his arm was touching her hand by chance, she withdrew her hand some millimetres in a nearly not recognizable gesture. On the other hand, they were talking about their spending a lot of time together and finally, about the way she was influenced by others and influenced him, telling him not to spend too much time with her so as not to be decreasing his own abilites. The crazy thing about her saying this was the fact, that she didn't use an ironic tone or something like, but was completely serious, probably even more than O. would be when being sarcastic. As a conclusion, he was left thinking that this was just a friendship; however, several things she did and several words she chose assumed something different, while the only other conclusion was, that they all thought he was a homosexual.
He wouldn't accept this, not after having written such a story for the sake of a lost love. 

Things were always going wrong, and one could just help oneself to endure; changing something that was changing itself wouldn't produce anything that would last. 
He felt ill at ease, when P.'s friend was waving at him, sometimes running up to him to greet him warmly, as he didn't know the final destination; nevertheless, he decided to enjoy it, ignoring that feeling. 
The only thing he would lose was time, and time was at the same time the only thing one owned and could never control.

He would feel in a way quite similar to the atmosphere of somebody sitting in a warm chamber, next to a crackling fire, sleeping or looking around drowsily, probably reading a book or slowly conforting a cat to comfort him- / herself when he was together with one of those friends of his.

This was the way he wanted to be, though he'd probably not like to be alone, at least, not all of the time. There were more things wating outside, but for now, his time was limited once more; probably, he'd be able to go into detail tomorrow, if more was to happen then. 
I'd be glad if you'd join me again when the next text arrives\dots{}
Have you made up your mind and come to some conclusion?

\begin{quote}
Crackling \\
was the logic \\
of reality. \\
Crackling \\
was the perspective \\
he'd found \\
to see the world. \\
And crackling \\
was the emotion \\
that had been subdued. \\
Crackling like a fire, \\
once burning bright and powerful, \\
then burning low; \\
who'd predict \\
whether --- and when --- \\
it would've consumed \\
itself? \\
--- W.G.
\end{quote}

\begin{quote}
Losing control \\
is the basis of realizing \\
you'd never been able to control a thing. \\
Regaining it --- \\
or never losing it --- \\
is the first step \\
to the end of existence. \\
--- W.G.
\end{quote}
