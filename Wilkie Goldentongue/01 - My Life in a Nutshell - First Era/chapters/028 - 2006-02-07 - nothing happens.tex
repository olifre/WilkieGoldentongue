\chapter{Nothing happens}
\label{cha:nothing-happens}
\subsection*{Originally published: \DTMDate{2006-02-07}}
\begin{quote}
It's me --- again. 

Tuesday: A day when he was bound not to see her, most probably. 
But now, he was indifferent to that. 
What had happened today?
Is the title correct, did nothing happen at all?

Find out: Nothing happens.
\end{quote}

That morning something happened. 
He felt sick, and somehow, his body must've thought that he was in need of showing this to his environment. Something strange happened, and he \emph{knew} that it was some sort of sign. 
It showed that he was ill, nothing more --- there's no need to go into detail now. 
And when the afternoon would arrive, he would be nearly completely healed again.

His mother was amazed when she saw that process getting hold of her son, but he knew that the power of the soul is the major power of our existence --- our physical power is nothing compared to it. 
He was sure he wouldn't talk to O. that day, but some communication took place, nevertheless.

He saw her from a distance of two or three metres away, and at first, he thought that she was the girl that had sat next to him on Friday; but just for a short moment. And then, he could see her in full detail, though she was too far away: Her silver necklace, which had the form of her favourite animal.

He realized he could even remember the name of the probably only individual of that kind she'd ever touched. 
But he saw nothing more; just her hair that was fluttering around her head. Then, she was gone. He knew he would have to head the same way, too, but he'd have to wait for P., who called after him. And when he finally had gone round the corner, she was vanished.

A couple of minutes later, he stood in the hallway, waiting for the lecture to begin; He knew \emph{exactly} that O. would have passed here, too. 
Well, she hadn't, as suddenly, she passed here. He stared, and smiled. She just answered with her sort of smile that leaves you thinking it was just meant for you --- the sort of smile that jumps all over her face and resolves into a trace of happiness. 
An attracting smile\dots{}

His fellow students wished her a nice lecture, something which happened quite rarely; but the last weeks, when she'd passed here at exactly the same time, he'd done that. 
This time, he didn't, but the others did.

Amazing; the world would never stop changing. 
However, he wished he'd done it, though he was pretty sure he was \emph{done} with her. 

When the lecture he would have attended was finally cancelled, he walked along the hallway, alone, watching the pictures of all those students in the different lectures. 
Of course, he was among them. 
And she was, too. 

But before he'd go and examine the image that showed just the members of \enquote{her} lecture, he would have a look at the old image\dots{}
He'd never thought of it again, though he owned it. 
But it was four years old, and at that time, he hadn't really \emph{known} neither O. nor G. 
Well, now he did, and he searched for them --- and for himself, of course.

He found G. quickly, as she was standing next to Y., her best friend at that time; but as the two of them had been together for too long a time, they were now going separate ways --- kind of. They were still friends, that is, but just in a way of comradeship, nothing more. 
He still hadn't found himself in the mass of students. 
Then, he made out the person some centimetres below G.; that was him. 
But that one must be another picture, not the one he had at home, as it showed his shirt. In the copy he had, only his head could be seen\dots{}
Or was it finally a different person?

He stopped thinking about it, and started searching for O. again. He was pretty sure that she was standing quite near to G. He thought he'd recognized O.'s best friend just behind G., but as he didn't even know if she was studying at that university at that time, he was stupefied. And, of course, at that time, she had not been O.'s best friend, and he hadn't even known her name. 
Then, he found her --- probably. He wasn't sure. When he looked at her, she looked quite the same as G. Crazy. He went so close to the picture that he had to bow, and the tip of his nose nearly touched the plastic in front of it.

He wasn't sure, so he started searching again, his eyes following the lines of students, hundreds of them. 
But he didn't find another girl that resembled her that well. And her hair had been different, more like G.'s\dots{}
The picture was smaller than his pinkie, maybe just the half of it's size. 
He went even closer.

Now, some features stood out more clearly; but still, this could just be his imagination that \emph{wished} to have found her. Then, he went over to the bigger picture, which showed her, sitting leisurely next to B.-B. 
Her best friend sat next to B.-B., and he was standing some steps behind them. In that image, he could make her out clearly, as it wasn't four years old; and, in addition to that, her face was even bigger than his thumb. 
He wondered if she would have chosen some place nearer to him, if she hadn't arrived so late. All the students of that lecture had been waiting for her\dots{}\\
\textbf{FLASHBACK}\\
He was in the bus, G. was seated behind him, while O. was somewhere farther away. At that time, he was still interested in G. 
And she was thinking about some really easy song: 
\begin{quote}
  \enquote{I'm a big, big girl in a big, big world and it's no big, big thing if you leave me.......}.
\end{quote}
Talking about it to nobody, in fact, she just catched the sense of that sentence. 
He was happy that she thought about such things, and tried to interpret them; but something inside him realized that this was not the kind of women he'd like. She was too shallow, not thinking about more complicated things. But she had the interest to do so, and if one would try to help her\dots{}\\
O. was even more interested, he was pretty sure of it\dots{}\\
\textbf{END}

He was back to real life again, walking once more next to the picture that showed the mass of students. 
He'd accept that he wouldn't be sure if that was her, and was gone soon, heading for the library. There, he joined some of his friends. 
When the next lecture was finished, he would have to head for the bus. 
And suddenly, he recognized O. when she arrived from a direction that made no sense --- in the beginning. Then, he realized that this was a shorter path to return from the auditorium she had been in to the library, or some other more centralised room.

He'd used that cutoff quite often, recently. 

Well, she passed him, without him turning his head, as he tried to stay fixed on the screen in front of him. 
And he succeeded, while he would stand up some seconds later to catch the bus. When he passed O., who was talking to somebody sitting there, he said "Goodbye!", not using their special way of saying it, as there were a whole bunch of people around. 
She seemed not to hear him.

Should he have added her name to that word?
Probably. But that told him something: If you really like somebody, or even love that person, you've got a mental picture of that individual, which means, that you recognise all the noises, the voice, the scent, the perfume, the skin, the feeling, the eyes and the belongings of that person in an instance. 
He did, at least. 
And as she didn't, this was the answer to his question. 
However, he felt that this was quite an instable theory; He'd have to prove it several times. 
Then, he learned from P. that her friend --- the one we've been talking about so often now --- had prepared something she'd shown him. That meant that she'd planned to do so some time before --- and as we know, plans are never realised to our benefit.

Thus, he knew that if he'd offended her, she must've waited for some happy emotions. And he hadn't really recognised what she'd done\dots{}
Well, it was too late now to do something about that. If fate wishes to lead us somewhere, it will offer us several chances --- most probably. 
Will there be another chance tomorrow, or in the upcoming week?
He wondered, why he'd met O. so rarely that week. 
He knew the reason why.

The world was a rhythm. 
We'd already talked about the sum of everything being zero, finally. If we assume the happenings that approach us being a sine, we can compare the happy things --- the area above zero --- with the not so happy things --- the area below zero. 
You'll find that both are equally high, and always tend to reach an equilibrium. 

Thus, he'd been together with her the last week for quite a long time, and this week, he had to pay for this, in some way. He thought back, what had happened last year\dots{}
Exactly one year and one week ago, he'd been together with her --- for a long time. Only then, she'd been single\dots{}
That was gone, now.

After that week, what had happened?
For some time, everything around her was silent, and then, they'd found together even more often. 
But he didn't dare hope for this cycle to repeat.

Then, he remembered something a professor had written in a book about sleeping: There was a cycle of about four hour's length, one of a day's length, one of a week's and one of a month's --- and one of a year's, of course. 
Maybe, even more. 
Though that book had been quite old, he knew that this man was right. 
Nobody would ever prove him wrong, without being wrong himself.

That meant, that several cycles were to meet: The cycle of her being single, the cycle of the beginning of the year, and probably even more\dots{}
Life was probably too complicated to grab all it's details mathematically.

Yin and Yang --- the areas above and below the sine\dots{}
And the basic frequencies in which the strings that made our atoms would vibrate, thus forming the giant harmony of life. 
Everything was based on a wave, finally, and the only thing most people didn't know, was that the wave would rise equally high as it would then fall low.

He felt he'd discovered something important. 
But finally, nothing had really \emph{happened}. 
Thus, we'll have to wait 'till another post arrives --- or 'till fate interferes again. 
Now, there will probably be some longer break, but don't be afraid --- this story will continue soon. 
Please stay tuned\dots{}
If you tell me your opinions, I'll probably try to continue earlier!

\begin{quote}
The snow was falling down, \\
hiding the innocent \\
and the guilty. \\
Everything was white, \\
everything was equal. \\
Everything \emph{seemed} equal. \\
Everything \emph{was} different. \\
But in the end, it would all sum up \\
to nothing. \\
--- W.G.
\end{quote}

\begin{quote}
Power \\
is the basis of our existence. \\
Why do we fail to recognise \\
that weakness is of equal importance? \\
--- W.G.
\end{quote}
